\section{Detailed Description of the Functions}\label{sec:detail}

So, you have read through section~\ref{sec:quickstart}, everything
worked fine, you can control you radio with grig and you are now ready
to learn more about grig. This section will give you a detailed
description of each high level function in grig so that you can learn
how to use them. By high level function I mean a functionality in grig
that is visible to you, the user. Examples of high level functions are
the software memory and the band map. If you feel that you already know
how to use grig, you can skip this section and go ahead with
section~\ref{sec:advanced}, which will tell you about what is going on
behind the scenes.



\subsection{Radio Info}


\subsection{Additional Controls}

The basic user interface in grig gives you imediate access to the basic
controls that are supported by most radios. These include frequency,
RIT/XIT, communication mode and similar. Many newer radios, however,
allow remote control of many more parameters than those, which are
visible in the basic user interface of grig. These parameters can be
accessed via the View menu item of grig.

\subsubsection{Receiver Level Controls}

\subsubsection{Transmitter Level Controls}

\subsubsection{CW Related Controls}

\subsubsection{DCS and CTCSS Tone Controls}

\subsubsection{Special Functions}


\subsection{Saving and Loading the Radio State}


\subsection{Software Memory}


\subsection{The Band Map}

\subsubsection{Basic Use of the Band Map}

\subsubsection{Special Features of the Band Map}


\subsection{Spectrum Scope}
